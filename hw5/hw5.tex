\documentclass{article}
\usepackage[margin=1.25in]{geometry}
\usepackage{amsmath, amssymb, setspace, enumerate, enumitem}
\usepackage{setspace}
\usepackage{graphicx}
\onehalfspacing

\begin{document}
    \begin{enumerate}
        \item Exercise 2.8
        \begin{enumerate}[label=(\alph*)]
            \item We know that $\overline{g}$ is the average function of many different hypothesis $g_1,\ g_2,...,\ g_n$ of different data sets. $H$ represents the hypotheses set, where each hypothesis in $H$ is dependent on their respective data set. If the hypothesis in $H$ are in linear combination, then the average of the hypotheses in $H$ should also be a linear combination, proving that $\overline{g} \in H \hfill \blacksquare$
            \item We can imagine a model with two hypotheses, one that will label all datapoints as $+1$ and another as $-1$, then the average of those two hypotheses will be $0$, which is not in the hypothesis set, therefore $\overline{g} \notin H \hfill \blacksquare$
            \item (b) is a binary classification and $\overline{g}$ is not a binary function since $0 \notin \{+1, -1\}$. Often, with more hypotheses, the average will be a number between $\{-1, 1\}$, it will be unlikely that they are $+1$ or $-1$
        \end{enumerate}

        \item Problem 2.14
        \begin{enumerate}
            \item Given that the $d_{vc}$ is finite, we know that the hypothesis can shatter any data set of size $d_{vc}$. We can claim that $K(d_{vc}) \geq d_{vc}(H)$, since $K(d_{vc})$ assumes that every $K$ hypothesis can shatter the maximum number of points, $d_{vc}$. Then $K(d_{vc} + 1) > K(d_{vc}) \geq d_{vc}(H)$, by transitivity, $K(d_{vc} + 1) > d_{vc}(H) \hfill \blacksquare$.
            \item 
        \end{enumerate}
    \end{enumerate}
\end{document}